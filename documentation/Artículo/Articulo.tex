\documentclass[conference]{IEEEtran}
\usepackage[utf8]{inputenc}
\usepackage[spanish]{babel}
\usepackage{amsmath}
\usepackage{amsfonts}
\usepackage{amssymb}
\usepackage{graphicx}
\usepackage[left=2cm,right=2cm,top=2cm,bottom=2cm]{geometry}
\author{\IEEEauthorblockN{Paul Percca, Ivan Sipiran} 
\IEEEauthorblockA{Pontificia Universidad Católica del Perú}
\IEEEauthorblockA{Lima, Perú}
\IEEEauthorblockA{cristian.percca@pucp.edu.pe, isipiran@pucp.edu.pe}
}
\title{Identificación Automática del Comportamiento de Clientes de una Tienda Retail mediante Secuencias de Video utilizando Aprendizaje Profundo}

\begin{document}

\begin{figure}[hbtp]
\centering
\includegraphics[scale=0.9]{Figuras/reporte.png}
\end{figure}

\maketitle

\begin{abstract}
Las tiendas físicas usan múltiples técnicas para analizar el comportamiento del cliente, la mayoría de estas, se centra en la captura de datos en caja, al finalizar el proceso de compra \cite{boucouvalas2015integrating}. Por lo tanto, se pierde información valiosa sobre el comportamiento del cliente, durante su estadía en la tienda. Conocer la satisfacción de los clientes, es extremadamente valioso para las tiendas \cite{karim2018customer}.
Conocer las rutas de movimiento que tomaron los clientes dentro de una tienda y conocer el grado de interés de los clientes frente a algún producto, es importante para los gerentes de tienda, ya que, esos conocimientos puede ayudar a crear mejores campañas de marketing y publicidad dirigida, así como optimizar el almacenamiento y la operación de la tienda \cite{yaeli2014understanding}. Este artículo se enfoca en análisis de video en una tienda retail con el fin de identificar las zonas de la tienda con mayor y menor tráfico así como identificar las zonas con mayor demanda de consumo utlizando técnicas del estado del arte en deep learning como son object detection (YOLOv3) y pose estimation (Deep Pose).

\end{abstract}

\begin{IEEEkeywords}
Object Detection, YOLOv3, Pose Estimation, Deep Pose
\end{IEEEkeywords}

\section{Introducción}
Las tiendas retail, realizan una amplia variedad de actividades, muchas de las cuales pueden ser asistidas por análisis de video, como la planificación de diseños de tiendas basadas en estadísticas de ruta de clientes, el recuento de clientes histórico e instantáneo, entradas y salidas de tiendas, colas de pago, entre otras. Cada transacción comercial y paso durante el proceso de compra, genera una gran cantidad de datos \cite{provost2013data}, que pueden generar indicadores para la toma de decisiones.

Entre esas actividades se encuentra la prevención de pérdidas y la medición del nivel de satisfacción de sus clientes. La prevención de pérdidas se clasifica en cuatro grupos primarios: pérdidas internas (trabajadores de la tienda), externas (supuestos clientes, mediante cambio de etiquetas, fraude en la devolución de productos, entre otros), proveedores y administrativas \cite{deyle2015global}.

Por otro lado, muchos clientes se sienten frustrados debido a que no encuentran su producto deseado, ya sea por haber sido colocados en góndolas inadecuadas o en zonas de la tienda que no son muy transitadas, esto generan una mala experiencia al cliente, lo cual se traduce en pérdidas de venta. Si bien se realizan estrategias comerciales y operativas para el posicionamiento de los productos, y estudios de mercado para ello, esta información no se obtiene en tiempo real y muchas veces quedan obsoletas al ser aplicadas.

Hoy en día existen varias soluciones para análisis de datos de e-commerces, sin embargo, no se tiene mucha información de lo que sucede en la tienda física, por lo tanto, esto puede resultar un inconveniente para los administradores de la tienda; mediante el uso de análisis de video, aprendizaje profundo y procesamiento de imágenes. se tiene una recopilación más activa de la información \cite{karim2018customer}, información como, cuántos clientes visitan la tienda, cuántos de las personas que entraron a la tienda compraron, cuánto tiempo permaneció una persona en la tienda, cuál fue la trayectoria de un cliente en la tienda, o cuáles son los productos donde la gente se detiene más o menos tiempo a observar, identificar el interés mediante los gestos que muestra un cliente al estar parado frente a un producto específico, es una información muy importante para la toma decisiones, ya que  se puede tener una visión general y tomar esta información para realizar campañas de marketing o reorganizar la ubicación de los productos de ser necesario.

El enfoque de este trabajo se centra identificar las zonas de la tienda con mayor y menor tráfico y así como identificar las zonas con mayor demanda de consumo utlizando técnicas del estado del arte en deep learning como son object detection (YOLOv3) y pose estimation (Deep Pose) esta información va a ser mostrada en heatmaps de manera que pueda ser de mayor provecho para los gerentes de tienda y así tomar decisiones que ayuden a la recolocación de productos con el fin de aumentar las ventas, así mismo este análisis se podrá realizar a demanda pudiento conocer el comportamiento de los clientes a cuando se requiera.

\section{Antecedentes Teóricos}

\subsection{Redes Neuronales Artificiales (ANN)}
Las redes neuronales artificiales son un paradigma de programación inspirado en la biología que permite a un algoritmo aprender a partir de un conjunto de datos observacionales \cite{nielsen2018neural}. La investigación en este campo ha tenido grandes avances, desde que McCulloch y Pitts propusieron su modelo de neurona que procesaba cada entrada multiplicada por un peso específico,para luego sumarlas y obtener una salida de 0 y 1 \cite{mcculloch1943logical} como se aprecia en la Figura \ref{fig:perceptron} tomado de \cite{rojas2013neural}  hasta la actualidad donde se cuenta con Deep Learning.

\begin{figure}[hbtp]
\centering
\includegraphics[width=8cm]{Figuras/perceptron.png}
\caption{Modelo de una neurona artificial}
\label{fig:perceptron}
\end{figure}

\subsection{Deep Learning}

Deep Learning permite a modelos computacionales compuestos, por multiples capas de procesamiento, aprender representaciones más complejas con múltiples niveles de abstracción, y todo esto gracias al uso de un algoritmo de retropropagación que permite ir ajustando los pesos hasta las capas anteriores \cite{lecun2015deep}, es decir, Deep Learning es un poderoso conjunto de técnicas para aprender en redes neuronales que proveen las mejores soluciones para muchos problemas en diferentes áreas como reconocimiento de imágenes, reconocimiento de voz y procesamiento de lenguaje natural \cite{nielsen2018neural}. A continuación se muestra un esquema de un modelo de Deep Learning en comparación de una Red Neuronal.

\begin{figure}[hbtp]
\centering
\includegraphics[width=8cm]{Figuras/deeplearning.png}
\caption{Modelo de un Red Neuronal Profunda (Deep Learning)}
\label{fig:deeplearning}
\end{figure}

\subsection{Redes Neuronales Convolucionales (CNN)}
Las Redes Neuronales Convolucionales son un tipo de redes neuronales ...

\subsection{You Only Look Once (YOLO)}

YOLO es un modelo de Red Neuronal Convolucional usada para la detección de objetos que predice simultáneamente múltiples cuadros delimitadores y probabilidades de clase para esos cuadros, si bien YOLO comete más errores de localización, es menos probable que prediga falsos positivos, además de ello es extremadamente rápido, pudiendo correr a 45 fps en su versión base y en una versión fast a 150 fps, y esto debido a que se plantea la detección como un problema de regresión, YOLO ve la imagen completa, codifica mejor la información contextual sobre las clases y su apariencia a diferencia de otros modelos basados en propuestas de ventana deslizante y región. aprende representaciones generalizables de los objetos, al ser entrado con imágenes naturales y luego probado con ilustraciones, supera otros métodos como DPM y R-CNN \cite{redmon2018yolov3}. La arquitectura de este modelo se muestra en la Figura \ref{fig:yolo}

\begin{figure}[hbtp]
\centering
\includegraphics[width=8cm]{Figuras/yolo.png}
\caption{Arquitectura de YOLO}
\label{fig:yolo}
\end{figure}

YOLO divide cada imagen en una cuadrícula de S x S y cada cuadrícula predice B cuadros delimitadores y puntuaciones de confianza, que reflejan la confianza de que el cuadro contiene un objeto y la precisión con la que considera que el cuadro predice, de no haber una imagen en dicha celda, el puntaje es cero. Cada cuadro limitador realiza 5 predicciones x, y, w, h y el puntaje de confianza, donde (x, y) es el centro del cuadro, “w” y “h” son el ancho y el alto que se del objeto que se predice \cite{redmon2018yolov3}.


\subsection{Pose Estimation}
...

\section{Trabajos Relacionados}
El análisis de video en el campo de retail, es un tema relevante para los investigadores, incluso grandes  empresas han realizado investigaciones en este rubro:

En \cite{ellis2002performance} se realizó una investigación en procesamiento de video en un contexto retail, basándose en un video grabado en un centro comercial, el objetivo de esta investifación era contar las personas que pasaban y saber cuántas se quedaban frente a un escaparate. De igual manera en \cite{haritaoglu2001detection} se centraron en determinar "grupos de compras" que esperaban en la cola de pago y en \cite{leykin2007detecting} utilizaron algoritmos de enjambre para agrupar clientes "grupos de compras", cabe mencionar que "grupo de compras" es un grupo de persoans que realizarían una compra conjunta, por lo tanto manejar el tráfico del grupo sería un mejor indicativo.

En \cite{senior2007video} IBM planteó un sistema de vigilancia y análisis de video que se centró en la prevención de pérdidas, permitiendo al usuario elegir regiones activas y observar cuántos clientes ingresaron a una región en un período de tiempo, cuántos se detuvieron allí y cuánto tiempo pasaron estos clientes. Se muestran todas las trayectorias de los clientes y permiten al usuario "profundizar" en el video original para observar el comportamiento de los clientes seleccionados. Esta solución consistía en realizar el seguimiento de objetos genéricos y el seguimiento de caras, dicha información de apariencia, trayectoria y fotograma clave del objeto se enviaba a la base de datos, para realizar la clasificación de objetos usaron AdaBoost. Para esta investigación se usaron 6 cámaras, servidores duales Pentium de 3.6GHz para el análisis de video, administración de video y MILS. El algoritmo de seguimiento de ColourField se ejecutaba a 30 fps. La sustracción de fondo toma entre 5,5 y 8,5 ms por fotograma y el seguimiento tomaba entre 2 y 4 ms cuando había un primer plano que se debía seguir.

En la detección de personas también se han realizado investigaciones como:
En \cite{gajjar2017human} donde se realiza una investación en la detección y seguimiento de humanos para videovigilancia con los datos de la Universidad Estatal de Ohio y se incorpora histogramas de gradientes orientados (HOG), la teoría de la prominencia visual y el modelo de predicción de prominencia Deep Multi-Level Network, se usó las funciones HGO para obtener una máquina de vectores de soporte (SVM) para detectar humanos en cualquier frame, para el seguimiento de personas se usó del algoritmo k - means que permitió encontrar los patrones de movimiento en los frames, lo cual es relevante para la reidentificación de la persona. además de ello se usó el algoritmo k - means para comparar y agrupar las características HOG de cuadros consecutivos para identificar el conjunto de puntos en la imagen que se asemejan a una persona en particular que se mueve en el video.

\section{Diseño del Experimento}
A continuación se explicará de forma detallada el diseño de la solución presentada en este trabajo de investigación:

Se han usado n secuencias de video  de 720 px por 1280 px con una duración total de m horas para el presente trabajo de investigación tomadas de ........

Debido a que se está trabajando con videos, se ha realizado un preprocesamiento, que consiste básicamente en:
\begin{itemize}
\item Remover el audio de los videos.
\item Dividir el video en frames y etiquetarlos.
\end{itemize}

Debido a que la detección de personas es un problema conocido se optó por un modelo de YOLO v3 (Darknet) preentrenado con el dataset PASCAL que inicialmente detectaba 40 clases entre las cuales se encontraba la clase Persona, se realizó Transfer Learning con el dataset Y mejorando el accuracy al detectar solo una clase.



\section{Experimentación y Resultados}

\section{Discusión}

\section{Conclusiones y Trabajos Futuros}

Como trabajos futuros, se plantea aumentar más indicadores con información relevante para la toma de decisiones, así como la implementación de otros modelos de red neuronal, cuyo foco sea la seguridad y la detección de fraudes en las tiendas retail.

\section*{Agradecimientos}

\bibliographystyle{plain}
\bibliography{Bibliografia}

\end{document}



\begin{comment}

(pdf)latex + bibtex + (pdf)latex + (pdf)latex [+ preview]

[3] Ross Girshick, Forrest Iandola, Trevor Darrell y Jitendra Malik,
Deformable Part Models are Convolutional Neural Networks2015
. pp 437-439.
[4] P. F. Felzenszwalb, R. B. Girshick, D. McAllester, and D. Ramana,

\end{comment}